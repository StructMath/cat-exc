It is trivial that in \textbf{Set} the bijections are isomorphisms. For the converse, assume that for $f:A\to B$ and $g:B\to A$, we have $gf=id_A$ and $fg=id_B$. The morphism $f$ must be surjective. If not there is a $b\in B$ that is not in the range of $f$. Such an $f$ cannot be in the range of $fg$ too, and hence $fg\neq id_B$. It also needs to be injective. If not there are $a,a'\in A$ such that $a\neq a'$ and $f(a)=f(a')$. Then we have $gf(a)=gf(a')$, and hence $gf\neq id_A$.

In a poset the only isomorphisms are the identities, since between any two objects there is at most one morphism. In a monoid any element with an inverse is an isomorphism. in $\textbf{Set}^\mathcal{C}$
